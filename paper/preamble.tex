%--------------------
% PACKAGES
\usepackage[utf8]{inputenc} % allow utf-8 input
\usepackage[T1]{fontenc}    % use 8-bit T1 fonts

\usepackage[pagebackref]{hyperref} % hyperlinks
\usepackage{url}            % simple URL typesetting

\usepackage[dvipsnames,table]{xcolor} % colors

\usepackage{graphicx}       % Required for inserting images
\usepackage{booktabs}       % professional-quality tables
\usepackage{multicol}       % cells spanning multiple columns
\usepackage{multirow}       % cells spanning multiple rows
\usepackage{makecell}       % multiline cells, etc
\usepackage{colortbl}       % colours in tables
\usepackage{caption}        % caption package
\usepackage{subcaption}     % make subfigures with their own graphics and own captions (a), (b), etc.
\usepackage{wrapfig}        % can make a figure only use part of the column width, text wraps around it
\usepackage{adjustbox}      % nice package for shrinking too-wide tables down (cleaner than using \resizebox)
\usepackage{titletoc}       % table-of-contents management, allowing a TOC for just appendix matter

\usepackage{mathtools}
\usepackage{amsmath,amsthm} % nice mathematics display
\usepackage{amsfonts}       % blackboard math symbols
\usepackage{amssymb}        % for the checkmark symbol
\usepackage{nicefrac}       % compact symbols for 1/2, etc.
\usepackage{microtype}      % microtypography

\usepackage{algorithm,algorithmicx,algpseudocode}

%\usepackage{pdflscape}      % support for landscape PDF pages (rotated pages)
\usepackage{comment}        % provides comment environment (use to suppress LaTeX)
\usepackage{array}
% \usepackage{times}         % Set default font to Adobe Times Roman

\usepackage{doi}            % provides nice \doi command
\usepackage[normalem]{ulem} % provides strikeout with \sout
\usepackage{placeins}       % provides \FloatBarrier
\usepackage[detect-all]{siunitx} % provides \num, \SI and other SI unit commands

% Configure packages
%~~~~~~~~~~
% Configure captions to be small, with label in italic
\captionsetup{font=small,labelfont=it}
% You can configure DOIs to be in smallcaps with a thin space, instead of lowercase without a space
% \renewcommand{\doitext}{{\scshape doi:\,}}

%--------------------
% SYMBOLS

% For frozen/unfrozen icons
%
% If you are not using the "unfrozen" icon, please comment out its code and delete file icons/SPB_flame.pdf
%
% Snowflake: U+008acd
\usepackage{bbding}         % provides some symbols
\definecolor{cold}{HTML}{008acd}
\def\frozen{\raisebox{-.15ex}{\color{cold}\SnowflakeChevron}}
\DeclareUnicodeCharacter{2744}{\frozen}% ❄, you can use the unicode in latex source, but it looks like * so don't - just use \frozen instead
%
% Fire: U+1F525
% https://commons.wikimedia.org/wiki/File:SPB_flame.svg
% CC BY-SA 4.0 Anton Landao, editted by Scott C. Lowe (change: colourized, #cd1600)
% \def\unfrozen{\raisebox{-.15ex}{\includegraphics[height=1em]{icons/SPB_flame.pdf}}}
\DeclareUnicodeCharacter{1F525}{\unfrozen}% ��, unicode symbol not currently supported on overleaf; just use \unfrozen instead

% Cross-mark (xmark) to use in tables like \checkmark
\usepackage{pifont}         % provides some symbols
\def\cmark{\ding{51}} % Alternative to \checkmark
\def\xmark{\ding{55}}
\definecolor{vlightgray}{gray}{0.87}
\def\myxmark{{\color{vlightgray}\xmark{}}}
\def\correctmark{{\color{ForestGreen}\cmark{}}}
\def\incorrectmark{{\color{red}\xmark{}}}

%--------------------
% COLOURS

% Tableau colors
\definecolor{tblue}{HTML}{1F77B4}
\definecolor{torange}{HTML}{FF7F0E}
\definecolor{tgreen}{HTML}{2CA02C}
\definecolor{tred}{HTML}{FF0000}

% Change links to be signified by text colour instead of in boxes
% We use colours that are similar to the default box colours by link type
%
% Colours based on Phelype Oleinik's colours
% https://tex.stackexchange.com/a/525297/108091
% but with modifications to those by Scott
\definecolor{linkcolor}{HTML}{991408}  % red
\definecolor{citecolor}{HTML}{2E7E2A}  % green
\definecolor{filecolor}{HTML}{131877}  % dark blue
\definecolor{menucolor}{HTML}{727500}  % yellow
\definecolor{runcolor} {HTML}{137776}  % teal
\definecolor{urlcolor} {HTML}{0a2bbf}  % blue
%
% Just comment out this \hypersetup command if you want to revert to boxes
\hypersetup{colorlinks=true,linkcolor=linkcolor,citecolor=citecolor,filecolor=filecolor,menucolor=menucolor,runcolor=runcolor,urlcolor=urlcolor}
%
% An alternative option is to make all links blue:
%\hypersetup{colorlinks=true,linkcolor=urlcolor,citecolor=urlcolor,filecolor=urlcolor,menucolor=urlcolor,runcolor=urlcolor,urlcolor=urlcolor}

%--------------------
% SPACE ADJUSTMENTS

% less space for equations
%\expandafter\def\expandafter\normalsize\expandafter{%
%    \normalsize%
%    \setlength\abovedisplayskip{2pt}%
%    \setlength\belowdisplayskip{2pt}%
%    \setlength\abovedisplayshortskip{-8pt}%
%    \setlength\belowdisplayshortskip{2pt}%
%}

%--------------------
% MACROS

% Referring to sections nicely
\renewcommand{\sectionautorefname}{Section}
\let\subsectionautorefname\sectionautorefname
\let\subsubsectionautorefname\sectionautorefname
\newcommand{\appref}[1]{\hyperref[#1]{Appendix~\ref*{#1}}}% For referencing subsections of the appendix

% Concise references
\def\Snospace~{\S{}}% Provides § without a space after it, so it butts up against section numbers
% Uncomment these commands to save space, or if you just like using the section symbol (§)
%\renewcommand*{\sectionautorefname}{\Snospace}
%\renewcommand*{\subsectionautorefname}{\Snospace}
%\renewcommand*{\subsubsectionautorefname}{\Snospace}
%\renewcommand*{\appref}[1]{\hyperref[#1]{App.~\ref*{#1}}}
%\renewcommand*{\figureautorefname}{Fig.}
%\renewcommand*{\equationautorefname}{Eq.}
%\renewcommand*{\tableautorefname}{Tab.}

% A paragraph command that is like a subsubsubsection
\newcommand{\mypara}[1]{\noindent\textbf{#1}}

% Writing links with the URL written out in a footnote
\newcommand{\hreffoot}[2]{\href{#1}{#2}\footnote{\url{#1}}}

% Results highlighting
\newcommand{\best}[1]{\textbf{#1}}
\newcommand{\secbest}[1]{\underline{#1}}
\newcommand{\runnerup}[1]{\secbest{#1}}

% Proofing
\newcommand{\hider}[1]{}
\newcommand{\TODO}[1]{\textcolor{red}{[\textbf{TODO}: #1]}}

\newtheorem{theorem}{Theorem}
\theoremstyle{definition}
\newtheorem{definition}{Definition}